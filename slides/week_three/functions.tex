\documentclass{beamer}
\usetheme{metropolis}
\usepackage{listings}
\usepackage{xcolor}

\title{Functions \& Libraries}
\author{Jannusch Bigge}
\date{14.11.2023}

\begin{document}
\begin{frame}
    \titlepage
\end{frame}

\section{Functions}

\begin{frame}{Memory}
    Different types of memory:
   \begin{description}
        \item<1->[HDD/SSD] Hard Disk Drive / Solid State Drive  
        \item<2->[RAM] Random Access Memory
        \item<3->[Cache] Small and fast memory
   \end{description}
\end{frame}

% \begin{frame}{Memory}
%     Different layout of memory for tasks:
%    \begin{description}
%         \item<1->[Heap] Memory for dynamic allocation k
%         \item<2->[Stack] Memory for function calls
%    \end{description}
%    \begin{alertblock}<3->{Not important in Python - the VM does all the work}
%    \end{alertblock}

% \end{frame}

% difference between == and equals() in python
\begin{frame}{Memory - Reference and copy}

    \textbf{Python} \hfill \textbf{presudo memory}
   
    \begin{columns}
        \begin{column}{0.49\textwidth}
            \begin{itemize}
                \item<1-> Reference: \textit{b = a}
                \item<3-> Copy: \textit{a = b.copy()}
            \end{itemize}
        \end{column}
        \begin{column}{.02\textwidth}
            \rule{.1mm}{0.7\textheight}
        \end{column}
        \begin{column}{0.49\textwidth}
            \begin{description}
                \item<2->[a (00)] 00 03
                \item<2->[b (01)] 00 (address of a)
                \item<4->[--]
                \item<4->[a (00)] 00 03
                \item<4->[b (01)] 00 03    
            \end{description}
        \end{column}

    \end{columns}

    
\end{frame}

% Scope
\begin{frame}{Scopes and Namespaces}

    \begin{alertblock}<1->{Using every name only ones could lead to problems.}
    \end{alertblock} \pause
    \textbf{Solution:}

    \begin{itemize}
        \item<3-> Namespaces
        \item<4-> Scopes 
    \end{itemize}
    
\end{frame}

\begin{frame}{Namespaces}
    Mapping from names to objects:\pause
    \begin{itemize}
        \item<2-> Built-in
        \item<3-> Global
        \item<4-> Enclosing
        \item<4-> Local
    \end{itemize}

\end{frame}

\begin{frame}[fragile]{Scope - Single definition}
    \begin{lstlisting}[language=Python]
    >>> x = 'global'
    >>> def foo():
    ...
    ...     def bar():
    ...         print(x)
    ...        
    ...     bar()
    >>> foo() 
    global
    \end{lstlisting}
\end{frame}

\begin{frame}[fragile]{Scope - Double definition}
    \begin{lstlisting}[language=Python]
    >>> x = 'global'
    >>> def foo():
    ...     x = 'enclosing'
    ...     def bar():
    ...         print(x)
    ...        
    ...     bar()
    >>> foo() 
    enclosing
    \end{lstlisting}
\end{frame}

\begin{frame}[fragile]{Scope - Triple definition}
    \begin{lstlisting}[language=Python]
    >>> x = 'global'
    >>> def foo():
    ...     x = 'enclosing'
    ...     def bar():
    ...         x = 'local'
    ...         print(x)
    ...        
    ...     bar()
    >>> foo() 
    local
    \end{lstlisting}
\end{frame}

\section{Control structures}

\begin{frame}[fragile]{Control structures}
    Only sequential execution is nice but sometimes we need more:
    \begin{columns}
        \begin{column}{0.49\textwidth}
            \textbf{6th fibonacci number:}
            \begin{lstlisting}[language=Python]
first = 0
second = 1
third = 1
fourth = 2
fifth = 3
sixth = 5
            \end{lstlisting}
            \onslide<2->\textbf{Problem:} What if we want the 100th fibonacci number?
        \end{column}
        \begin{column}{.02\textwidth}
            \rule{.1mm}{0.7\textheight}
        \end{column}
        \begin{column}{0.49\textwidth}
            \onslide<3->\textbf{Solution:} Control structures
            \begin{lstlisting}[language=Python]
a = 0
b = 1
c = 1
for i in range(100):
    a = b
    b = c
    c = a + b
            \end{lstlisting}
    
        \end{column}

    \end{columns}
    
\end{frame}

\begin{frame}[fragile]{for loop}
    Repeat something a given number of times.
    \begin{itemize}
        \item \textbf{for} \textit{variable} \textbf{in} \textit{iterable}:
        \item \textbf{for} \textit{variable} \textbf{in} \textit{range(start, stop, step)}:
    \end{itemize}\pause
    \textbf{Example:}
    \begin{lstlisting}[language=Python]
        a = 2
        for i in range(3):
            print(a + i)
    >>> 2
    >>> 3
    >>> 4
    \end{lstlisting}
    
\end{frame}

\begin{frame}[fragile]{if statement}
    Do something only if condition is true.
    \begin{itemize}
        \item \textbf{if} \textit{condition}:
    \end{itemize}\pause
    Types of conditions:
    \begin{itemize}
        \item \textbf{True} or \textbf{False}
        \item \textbf{==} or \textbf{!=}
        \item \textbf{and} or \textbf{or}
    \end{itemize}\pause
    \textbf{Example:}
    \begin{lstlisting}[language=Python]
    a = 2                  a = 2
    if a == 2:             if a == 2 and a != 3:
        print(a)               print('Not 3')
>>> 2                       >>> 'Not 3'
    \end{lstlisting}
    
\end{frame}

\section{Task}

\begin{frame}
    First Task:
    \begin{itemize}
        \item Just try the fibonacci number sequence for different numbers.
    \end{itemize}
    Second Task:
    \begin{enumerate}
        \item Calculate a baseline
        \begin{enumerate}
            \item Calculate the mean of each cluster
            \item Calculate the mean of all points
            \item Sort the unclassified points into the two clusters
        \end{enumerate}
        \item Clustering with Gram-Schmidt (Bonus Task)
        \begin{enumerate}
            \item Find the perpendicular vector to the line between (-2, 6) and (6, -2)
            \item Find the threshold \textbf{t} for $w^Tx<t$ when x is in class one. Note: $w^Tx = w \cdot x$
        \end{enumerate}
    \end{enumerate}
\end{frame}

\section{Next week: Using functions and libraries}

\end{document}