\documentclass{beamer}
\usetheme{metropolis}
\usepackage{listings}
\usepackage{xcolor}

\title{Functions \& Libraries}
\author{Jannusch Bigge}
\date{14.11.2023}

\begin{document}
\begin{frame}
    \titlepage
\end{frame}

\section{Functions}

\begin{frame}{Functions}
    Often we want to do the same thing multiple times.\\ \pause 
    But we don't want to write the same code multiple times. \pause
    \begin{alertblock}{Solution: Functions}
        \begin{itemize}
            \item<3-> Reusable code
            \item<4-> Easier to read
            \item<5-> Easier to debug
        \end{itemize}
    \end{alertblock}
\end{frame}

\begin{frame}[fragile]{Functions - Definition in Python}
    Definition:
    \begin{lstlisting}[backgroundcolor = \color{lightgray},language=Python]
    def function_name(arguments):
        # do something
        return something
    \end{lstlisting}
    \begin{itemize}
        \item<2-> \textbf{def} \textit{name}(\textit{arguments}): - Start of the function
        \item<3-> \textbf{return} \textit{something} - End of the function
    \end{itemize}
\end{frame}

\begin{frame}[fragile]{Functions - Calling}
    Define:
    \begin{lstlisting}[backgroundcolor = \color{lightgray},language=Python]
        def fibbonacci(number):
            a = 0
            ...
                b = a + b
            return a 
    \end{lstlisting}
    Calling the function:
    \begin{lstlisting}[backgroundcolor = \color{lightgray},language=Python]
        >>> result = fibbonacci(6)
        >>> print(result)
        8
    \end{lstlisting}

\end{frame}

\begin{frame}{Some examples}
    You allready know some functions:
    \begin{itemize}
        \item<1-> \textbf{print}(\textit{something})
        \item<2-> \textbf{len}(\textit{something})
        \item<3-> \textbf{range}(\textit{something})
        \item<4-> \textbf{input}(\textit{something})
    \end{itemize}
    
\end{frame}

\begin{frame}{Return values}
    You can return none, one or multiple values.\pause
    \begin{itemize}
        \item<2-> \textbf{None} - Nothing
        \item<3-> \textbf{return a} - One value
        \item<4-> \textbf{return a, b, c} - Multiple values
    \end{itemize}\pause
    How to access multiple return values?

\end{frame}

\begin{frame}{Tuples}
    Special data type in Python: \textbf{Tuple}\\ \pause
    Stores multiple values in one variable.\\ \pause
    \begin{itemize}
        \item Immutable
        \item Ordered
        \item Can contain multiple data types
    \end{itemize}
\end{frame}

\begin{frame}[fragile]{Tuples - Examples}
    \begin{lstlisting}[backgroundcolor = \color{lightgray},language=Python]
        >>> my_tuple = (1, 2, 3)
        >>> print(my_tuple)
        (1, 2, 3)
        >>> print(my_tuple[0])
        1
        >>> print(my_tuple[1])
        2
        >>> print(my_tuple[2])
        3
    \end{lstlisting}
    Stuff like \textbf{len()} and \textbf{for} works as expected.
\end{frame}

\begin{frame}{Return types}
    \textbf{What is returned by the function?}\\ \pause
    \begin{itemize}
        \item In python you are not forced to reveal that.\\ \pause
    \end{itemize}
    \textbf{Why? }\\ \pause
    \begin{itemize}
        \item Python is a so called dynamically typed language.\\ \pause
        \item You also don't have to specify the type of the arguments.\\ \pause
    \end{itemize}
    \textbf{Should I do it anyway?}\\ \pause
    \begin{itemize}
        \item There is a reason why Python has this built in functionality.\\ \pause
    \end{itemize}
\end{frame}

\begin{frame}[fragile]{Type definition}
    \textbf{Defining the return type:}\\ \pause
    \begin{lstlisting}[backgroundcolor = \color{lightgray},language=Python]
        def fibbonacci(number) -> int:
            a = 0
            ...
                b = a + b
            return a
    \end{lstlisting} \pause
    \textbf{Defining the argument type:}\\ \pause
    \begin{lstlisting}[backgroundcolor = \color{lightgray},language=Python]
        def fibbonacci(number: int):
            a = 0
            ...
                b = a + b
            return a
    \end{lstlisting}
\end{frame}

\begin{frame}[fragile]{Type definition}
    \textbf{I don't know the type:}\\ \pause
    \begin{lstlisting}[backgroundcolor = \color{lightgray},language=Python]
        print(type(something))
    \end{lstlisting}
    \pause
    \textbf{For more complex returns you can use the typing module:}\\ \pause
    \begin{itemize}
        \item You can use \textbf{typing.Any} to indicate that you don't know the type.\\ \pause
        \item You can use \textbf{typing.Union} to indicate that you don't know the type but it is one of the types you specified.\\ \pause
        \item You can use \textbf{typing.Optional} to indicate that the type is one of the types you specified or \textbf{None}.\\ \pause
    \end{itemize}
\end{frame}

\begin{frame}
    Now we know how to write and use functions.\\ \pause
    Let's start using code from other people.
\end{frame}

\section{Libraries}

\begin{frame}{Libraries}
    Many libraries solve a lot of problems.\\ \pause
    \begin{itemize}
        \item \textbf{math} - Math functions
        \item \textbf{secrets} - strong random numbers
        \item \textbf{numpy} - fast/complex math
        \item \textbf{matplotlib} - plotting
        \item \textbf{pandas} - data analysis
        \item \textbf{tensor-flow} - machine learning
    \end{itemize}
\end{frame}

\begin{frame}[fragile]{Libraries}
    To use a library you have to import it.\\ \pause
    \begin{lstlisting}[backgroundcolor = \color{lightgray},language=Python]
        import math
    \end{lstlisting}
    \pause
    Now you can use the functions from the library.\\ \pause
    \begin{lstlisting}[backgroundcolor = \color{lightgray},language=Python]
        print(math.sqrt(4))
        2.0
    \end{lstlisting}
    Sometimes you only want to import a single function.\\ \pause
    \begin{lstlisting}[backgroundcolor = \color{lightgray},language=Python]
        from math import sqrt
        print(sqrt(4))
        2.0
    \end{lstlisting}
\end{frame}

\begin{frame}{Documentation}
    In general you can find the documentation of a library on the internet.\\ \pause
    \begin{itemize}
        \item \textbf{math} - \href{https://docs.python.org/3/library/math.html}{https://docs.python.org/3/library/math.html}
        \item \textbf{tensor-flow} - \href{https://www.tensorflow.org/api\_docs/python/tf}{https://www.tensorflow.org/api\_docs/python/tf}
    \end{itemize}
\end{frame}



\begin{frame}{Libraries}
    \begin{alertblock}{pip}

    \end{alertblock}
    Some libraries are not installed by default.\\
    You have to install them first.\\ \pause    
    But we will talk about that next week.
\end{frame}

\section{Next week:\\
 More data types and a bigger task}

\end{document}