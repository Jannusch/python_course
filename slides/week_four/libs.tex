\documentclass{beamer}
\usetheme{metropolis}
\usepackage{listings}
\usepackage{xcolor}

\title{Data Types \& Libraries}
\author{Jannusch Bigge}
\date{21.11.2023}

\begin{document}
\begin{frame}
    \titlepage
\end{frame}

\section{Libraries}

\begin{frame}{Libraries}
    Libraries are collections of functions.\\
    They are not all included in Python by default.\\
    \pause
    $\rightarrow$ We have to install them.\\
\end{frame}

\begin{frame}[fragile]{Libraries - Background}
    \textbf{Python} as the interpreter needs to know were to find the code.\\
    \pause
    \begin{lstlisting}[backgroundcolor = \color{lightgray},language=bash]
        echo $PATH
        whereis python3
        python3 -v
        ls /usr/lib/python3.11/
    \end{lstlisting}
    \pause
    $\rightarrow$ installing by hand is not a good idea.\\
    \pause
    $\rightarrow$ installing in general not the best idea.
\end{frame}

\begin{frame}{Virtual env}
    Python module that creates "fake environments"\\
    \begin{itemize}
        \item Isolated from the rest of the system
        \item Can be deleted without any problems
        \item Can be shared with others
    \end{itemize}
    
\end{frame}

\begin{frame}[fragile]{Virtual env - activating}
    First we have to create a virtual environment. In this case called venv.\\
    \begin{lstlisting}[backgroundcolor = \color{lightgray},language=bash]
        python3 -m venv venv
    \end{lstlisting}
    Next we have to activate it.\\
    \begin{lstlisting}[backgroundcolor = \color{lightgray},language=bash]
        source venv/bin/activate
    \end{lstlisting}
    Now we can do our python stuff.\\
    \begin{lstlisting}[backgroundcolor = \color{lightgray},language=bash]
        ...
    \end{lstlisting}
    And finally if we are done we can deactivate it.\\
    \begin{lstlisting}[backgroundcolor = \color{lightgray},language=bash]
        deactivate
    \end{lstlisting}

\end{frame}

\begin{frame}[fragile]{pip}
    To install a python library we can use pip.\\
    \pause
    \begin{lstlisting}[backgroundcolor = \color{lightgray},language=bash]
        pip install numpy
    \end{lstlisting}
\end{frame}


\section{Data Types}

\begin{frame}{Data Types}
    Python has built in data types which allow us to store data.\\

    \begin{itemize}
        \item list $\rightarrow$ [1,2,3]
        \item tuple $\rightarrow$ (1,2,3)
        \item dict $\rightarrow$ \{1:2, 3:4\}
    \end{itemize}
    
\end{frame}

\begin{frame}[fragile]{List}
    Lists are the most common data type.\\\pause
    In contrast to tuples they are mutable.\\\pause
    Lists have a lot of helpful functions:\\
    \begin{lstlisting}[backgroundcolor = \color{lightgray},language=Python]
        my_list = [1,2,3]
        my_list.append(4)
        my_list[0] = 5
        my_list.remove(2)
        my_list.pop(0)
        my_list.insert(0, 1)
        my_list.sort()
        my_list.reverse()
    \end{lstlisting}

\end{frame}

\begin{frame}[fragile]{Dictonary}
    Dictonaries are a bit more complex.\\\pause
    They are a collection of key-value pairs.\\\pause
    They are mutable.\\\pause
    \begin{lstlisting}[backgroundcolor = \color{lightgray},language=Python]
        my_dict = {1:2, "three":4}
        my_dict[1] = 5
        my_dict["three"] = 6
        my_dict.pop(1)
        my_dict.popitem()
        my_dict.clear()
    \end{lstlisting}
\end{frame}

\section{Task}

\begin{frame}{Task - Advent of Code}
    Excellent way to practice your coding skills.\\
    \begin{itemize}
        \item Advent of Code is a yearly event
        \item 25 days of coding
        \item 2 problems per day
        \item 50 stars to collect
    \end{itemize}
    \begin{center}
        \url{https://adventofcode.com/}
    \end{center}
\end{frame}

\begin{frame}{Task - Advent of Code}
    \begin{alertblock}{The goal is:} \pause
        \begin{itemize}
            \item Understand the problem\pause
            \item Understand how to solve it with programming\pause
            \item Work with others together
        \end{itemize}
    \end{alertblock}
\end{frame}


\begin{frame}{Task - Advent of Code}
    \begin{itemize}
        \item Solve the first two problems of day 1 (2021)
        \item There is a test function in the repository.
        \item You have two data files. One for the test and one for the real problem.
        \item You can use any library you want.
        \item Solve the task in groups of two or three.
    \end{itemize}
\end{frame}


\end{document}

